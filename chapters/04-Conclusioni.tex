\documentclass[main.tex]{subfiles}

\begin{document}
\sloppy

\section{Conclusioni}%\label{sec:Conclusioni}
\vspace{1.0cm}
Inizialmente, SeismoCloud era un progetto volto a dare solo un \emph{early warning} nel caso si fosse verificato un terremoto; tuttavia, eventi recenti come il terremoto che ha colpito la Turchia e la Siria in data 6 febbraio 2023, hanno evidenziato il problema della ricerca e il soccorso dei superstiti, motivo per cui è nata l'idea di offrire i dati collezionati dai dispositivi ai soccorsi. Nel corso del lavoro, l’idea ha iniziato a prendere forma, utilizzando un modello di \emph{machine learning} in grado di riconoscere la presenza umana e catalogarla tramite audio; infatti, attraverso l'audio raccolto dal microfono dei dispositivi è possibile eseguire la rilevazione con risultati attendibili.\newline
Durante lo sviluppo embrionale del sistema, è stato necessario sincronizzarsi con gli altri membri del team per comprendere come far comunicare client e server, e viceversa; successivamente ognuno ha proseguito in autonomia lo sviluppo del proprio aspetto. \newline
L'obiettivo del mio lavoro è stata la progettazione e implementazione del sistema lato backend per gestire la \say{\emph{human presence detection}}, predisporre API per far comunicare il client con il server, creare un servizio per gestire, tramite notifiche, le comunicazioni del server con i client interessati ed infine progettare e implementare il database e le query per salvare e raccogliere dati. Il risultato è stato un prodotto efficiente e mantenibile, integrabile con il resto del sistema, che si presta ad essere il nuovo punto di partenza del progetto SeismoCloud.

\subsection{Sviluppi futuri}
Il sistema \say{\emph{post earthquake human presence detection}} è il nuovo punto di partenza che da vita nuova al progetto SeismoCloud. Infatti gli sviluppi futuri dell'applicazione e del sistema sono infiniti, tuttavia ne possiamo elencare alcuni:
\begin{itemize}
    \item Possibile collaborazione con il Corpo Nazionale dei Vigili del Fuoco per offrire i dati raccolti aiutando i soccorsi
    \item Utilizzo di un modello machine learning, per ascoltare l'ambiente circostante e catalogare il livello di pericolo in cui si potrebbe trovare una persona
    \item Ampliare il modello BNS attuale per quantificare le persone attraverso le voci
    \item Recuperare la posizione di un dispositivi in tempo reale solo nel momento in cui il dispositivo è in movimento
    \item Capire se un dispositivo si trova in un edificio che ha subito un collasso totale, parziale o nullo attraverso i dati raccolti all'avvenire di un terremoto
\end{itemize}


\end{document}